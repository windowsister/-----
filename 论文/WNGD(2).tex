%\documentclass[CJK,11pt,oneside]{cctart} % for Chinese input
\documentclass[a4paper,reqno,11pt]{amsart}



% Packages

\usepackage[left=1 in, right=1 in,top=1 in, bottom=1 in]{geometry}

\usepackage{amsfonts}
\usepackage{amssymb}
\usepackage{amsthm}
%\usepackage[tbtags]{amsmath}
\usepackage{amsmath}
%\usepackage{latexsym}
\usepackage{mathrsfs}
\usepackage{color}
%\usepackage{graphicx}

\usepackage{tikz} % Use TikZ for drawing
%\pgfrealjobname{WNGD}
\usetikzlibrary{calc}
\usetikzlibrary{intersections}

\usepackage{pdfsync} % enabling synchronization with pdf

\allowdisplaybreaks

%\usepackage[titletoc]{appendix}

%\usepackage{cases}


%\setlength{\oddsidemargin}{0mm} \setlength{\evensidemargin}{0mm}
%\setlength{\topmargin}{0mm} \setlength{\textheight}{220mm}
%\setlength{\textwidth}{155mm}

%\setlength{\oddsidemargin}{0mm} \setlength{\evensidemargin}{0mm}
%\setlength{\topmargin}{-20mm} \setlength{\textheight}{240mm}
%\setlength{\textwidth}{155mm}

\renewcommand{\theequation}{\thesection.\arabic{equation}} % the number for equations would be of form section.equation
\numberwithin{equation}{section} % reset the counter for equations at the beginning of every section

% Some Notations are defined here

% For mathrm
\newcommand{\TV}{{\mathrm{TV}_0^1}}
\newcommand{\TVchi}[1]{\mathrm{TV}_{\!\chi^N\!#1}}
\newcommand*\rmd{\mathop{}\!\mathrm{d}}
\newcommand{\sign}{{\mathrm{sign}}}
\newcommand{\rme}{{\mathrm{e}}}
\newcommand{\diag}{{ \mathrm{diag} }}
\newcommand{\Str}{\mathrm{Str}}

% For mathbb
\newcommand{\mbr}{\mathbb{R}}
%\newcommand{\mbt}{\mathbb{T}}
\newcommand{\mbs}{\mathbb{S}}
\newcommand{\mbn}{\mathbb{N}}

% For derivatives
\newcommand{\pt}{\partial_t}
\newcommand{\px}{\partial_x}

% For space
\newcommand{\hs}{{ \quad }} % for space in equations
\newcommand{\emptline}{$\phantom{A}$}

% For other symbols
\newcommand{\dm}{\diamondsuit}
\newcommand{\dmh}{\dm^N_{m,n}}
\newcommand{\dml}{\dm^N_{m,n,L}}
\newcommand{\dmr}{\dm^N_{m,n,R}}
\newcommand{\alphah}{\alpha^N_{1,m,n}}
\newcommand{\alphal}{\alpha^N_{1,m,n,L}}
\newcommand{\alphar}{\alpha^N_{1,m,n,R}}
\newcommand{\betah}{\beta^N_{1,m,n}}
\newcommand{\gammah}{\gamma^N_{1,m,n}}

\newcommand{\ts}{{\sigma}}
\newcommand{\tm}{\widetilde{M}}
\newcommand{\tme}{\widetilde{M}_E}
\newcommand{\vth}{\vartheta}
\newcommand{\hts}[1]{\hat{\sigma}_{#1}^N}

\newcommand{\dt}{\Delta t^N\!}
\newcommand{\dx}{\Delta x^N\!}

\newcommand{\TVm}{{\mathrm{TV}_{(m-1)\dx}^{(m+1)\dx}}}

\newcommand{\LR}{{\substack{L\\R}}}

\newcommand{\me}{M_E}
\newcommand{\ms}[1]{M_S(#1)}
\newcommand{\msn}[1]{M_S^N(#1)}
\newcommand{\mi}{\ms{0}}
\newcommand{\mz}{\ms{T_0}}
\newcommand{\mb}{M_*}

\newcommand{\smpsymb}{\bullet}
%\bigvarstar \star \bigstar


% For the formation of theorems

\newtheorem{lem}{Lemma}[section]
%\newtheorem{cor}[lem]{Corollary}
\newtheorem*{cor}{Corollary}
\newtheorem{prop}[lem]{Proposition}
\newtheorem{thm}{Theorem}
%\newtheorem{remark}[lem]{Remark}
\newtheorem{remark}[lem]{Remark}
%\newtheorem{thm_intro}{Theorem}
\newtheorem{fact}{Fact}
\newtheorem{claim}{Claim}
\newtheorem*{rem}{Remark:}

%\usepackage{multirow}




%%      ---------------------------------------------------------------------
%%      -------------------------- BEGIN ARTICLE ----------------------------
%%      ---------------------------------------------------------------------
                   %% Standard LaTeX command



\begin{document}

\title{Controllability to Systems of Quasilinear Wave Equations 
With Fewer Controls}


\author{Long Hu}
\address{The Institute of Mathematical Sciences\\
The Chinese University of Hong Kong\\
Shatin, NT, Hong Kong}
\email[Z. Xin]{zpxin@cuhk.edu.hk}
\thanks{Z. Xin is partially supported by the Zheng Ge Ru Foundation, Hong Kong RGC Earmarked Research Grants CUHK-14305315 and CUHK-4048/13P, a Focus Area Grant from the Chinese University of Hong Kong, and NSFC/RGC Joint Research Scheme N-CUHK443/14.}

\keywords{Weakly nonlinear gas dynamics; entropy solution; periodic initial datum; global existence.}

\subjclass[2010]{35L65,\ 35Q31,\ 35R09,\ 76N10.}

\author{Peng Qu}
\address{
School of Mathematical Sciences\\ Fudan University\\
Shanghai 200433, China
\newline \indent and
\newline \indent The Institute of Mathematical Sciences\\
The Chinese University of Hong Kong\\
Shatin, NT, Hong Kong}
\email[P. Qu]{pqu@fudan.edu.cn}
\thanks{P. Qu is partially supported by  National Natural Science Foundation of China (No.~11501121), Yang Fan Foundation of Shanghai on Science and Technology (No.~15YF1401100), Hong Kong RGC Earmarked Research Grants CUHK-14305315 and CUHK-4048/13P, Shanghai Key Laboratory for Contemporary Applied Mathematics at Fudan University and a startup funding of Fudan University.}

\author{Jiaxin Tong}
\address{The Institute of Mathematical Sciences\\
The Chinese University of Hong Kong\\
Shatin, NT, Hong Kong}
\email[Z. Xin]{zpxin@cuhk.edu.hk}
\thanks{Z. Xin is partially supported by the Zheng Ge Ru Foundation, Hong Kong RGC Earmarked Research Grants CUHK-14305315 and CUHK-4048/13P, a Focus Area Grant from the Chinese University of Hong Kong, and NSFC/RGC Joint Research Scheme N-CUHK443/14.}

\keywords{Weakly nonlinear gas dynamics; entropy solution; periodic initial datum; global existence.}

\subjclass[2010]{35L65,\ 35Q31,\ 35R09,\ 76N10.}

%\usepackage[pdftex]{hyperref}

\begin{abstract}
	This study addresses challenges in solving exact boundary controllability problems of quasilinear wave equations with Dirichlet type boundary conditions satisfying some kind of 
	Kalman rank conditions. Utilizing techniques such as characteristic method and a specially designed linearization iteration scheme, we developed controls that use fewer boundary 
	controls than the classical theory.
\end{abstract} % Abstract should precede \maketitle in AMS document classes

\maketitle

%\tableofcontents


%%%%%%%%%%%%%%%%%%%%%%%%%%%%%%%%%%%%%%%%%%%%%%%%%%%%%%%%%%%%%%%%%%%%%%%%%%%%%%%%%%%%%%%%%%%%%%%%%%%%%%
\section{Introduction}\label{intro sec}
%%%%%%%%%%%%%%%%%%%%%%%%%%%%%%%%%%%%%%%%%%%%%%%%%%%%%%%%%%%%%%%%%%%%%%%%%%%%%%%%%%%%%%%%%%%%%%%%%%%%%%
Consider the controllability for system of the quasilinear wave equations below
\begin{equation}\label{42}
    \begin{cases}
	U_{tt}-a^2\left( U,U_t,U_x \right) U_{xx}=0,&		\left( t,x \right) \in \mathbb{R}^+\times \left[ 0,L \right] ,\\
	t=0:\left( U,U_t \right) =\left( \varphi  \left( x \right) ,\psi \left( x \right) \right) ,&		x\in \left[ 0,L \right] ,\\
	x=L:U=0,&		t\in \mathbb{R}^+,\\
	x=0:U_t-D\left( U,U_t,U_x \right) U_x=H\left( t \right) ,&		t\in \mathbb{R}^+,\\
	t=T:\left( U,U_t \right) =\left( \varPhi   \left( x \right) ,\varPsi  \left( x \right) \right) ,&		x\in \left[ 0,L \right] .\\
\end{cases}
\end{equation}
where
$$
U=\left(u_1, \ldots, u_n\right)^T(t, x) \in C^2 .
$$
$a\left(U, U_t, U_x\right)>0$ is the $C^1$ uniform wave speed. $D\left(u_1, u_2, u_3\right)=\left(d_{i j}\left(u_1, u_2, u_3\right)\right)_{i, j=1}^n$ is an   $n \times n$ matrix with $C^1$ regularity. Part of the components of $H(t)=\left(h_1(t), \ldots, h_n(t)\right)^T$ would be chosen as boundary controls. We denote $D^{\pm}\left( U,U_t,U_x \right) =a I_n \pm D $ , which we request are invertible. We also request that
$$
\begin{aligned}
	\lVert \left( \varphi  \left( x \right) ,\psi \left( x \right) \right) \rVert _{\left( C^2\left[ 0,L \right] \right) ^n\times \left( C^1\left[ 0,L \right] \right) ^n}  < \varepsilon ,\\
	\lVert \left( \varphi  \left( x \right) ,\Psi \left( x \right) \right) \rVert _{\left( C^2\left[ 0,L \right] \right) ^n\times \left( C^1\left[ 0,L \right] \right) ^n}  < \varepsilon .\\
\end{aligned}
$$
where $0 < \varepsilon \ll 1$.
Without loss of generosity, we set $a(0,0,0)=1$, \ $H(t)=\left(h_1(t), \ldots, h_p(t), 0, \ldots, 0\right)^T$.Now we try to reduce the wave equation to a hyperbolic equation group. Denote
$$
\begin{aligned}
& \tilde{U}=D^{+}(0,0,0) U ,\\
& \tilde{V}^{\pm}=\partial_t \tilde{U} \mp \tilde{a} \left(\tilde{U}, \tilde{U}_t, \tilde{U}_x\right) \partial_x \tilde{U}.
\end{aligned}
$$
where
$$
\tilde{a} \left(\tilde{U} , \tilde{U} _t, \tilde{U} _x\right)=a\left(D^{+}(0,0,0)^{-1} U, D^{+}(0,0,0)^{-1} U_t, D^{+}(0,0,0)^{-1} U_x\right) ,
$$
$$
\tilde{D} ^{\pm}\left(\tilde{U} , \tilde{U} _t, \tilde{U} _x\right)=D^{\pm}\left(D^{+}(0,0,0)^{-1} U, D^{+}(0,0,0)^{-1} U_t, D^{+}(0,0,0)^{-1} U_x\right) .
$$
The original equations can be changed into
\begin{equation}\label{hyperbolic equations}
    \begin{cases}
    & \partial_t \tilde{V} ^{ \pm} \pm \tilde{a} \left(\tilde{U} , \tilde{U} _t, \tilde{U} _x\right) \partial_x \tilde{V} ^{ \pm}=\mp \partial_x \tilde{U} \left(\partial_t \pm \tilde{a} \left(\tilde{U} , \tilde{U} _t, \tilde{U} _x\right) \partial_x\right)\left(\tilde{a} \left(\tilde{U} , \tilde{U} _t, \tilde{U} _x\right)\right) ,\\
    & x=L:\tilde{ V} ^{-}=-\tilde{V} ^{+} ,\\
    & x=0: \tilde{D} ^{-}\left(\tilde{U} , \tilde{U} _t, \tilde{U} _x\right) \tilde{D} ^{+}(0,0,0)^{-1} \tilde{V} ^{-}=-\tilde{D} ^{+}\left(\tilde{U} , \tilde{U} _t, \tilde{U} _x\right) \tilde{D} ^{+}(0,0,0)^{-1} \tilde{V} ^{+}+\tilde{H} (t).
\end{cases}
\end{equation}
where $\tilde{H} (t)=2aH(t)$. Then the boundary condition can be written as
$$
\begin{aligned}
x=0: \widetilde{V}^{+}=B_0 \widetilde{V}^{-}+ & \widetilde{H}(t)-\left(\widetilde{D}^{-}\left(\widetilde{U}, \widetilde{U}_t, \widetilde{U}_x\right)-\widetilde{D}^{+}(0,0,0)\right) \widetilde{D}^{+}(0,0,0)^{-1} \widetilde{V}^{+} \\
& -\left(\widetilde{D}^{-}\left(\widetilde{U}, \widetilde{U}_t, \widetilde{U}_x\right)-\widetilde{D}^{-}(0,0,0)\right) \widetilde{D}^{+}(0,0,0)^{-1} \widetilde{V}^{-}.
\end{aligned}
$$
where the coupling matrix 
$$
B_0=-\left(a(0,0,0) I_n-D(0,0,0)\right)\left(a(0,0,0) I_n+D(0,0,0)\right)^{-1} .
$$
Define the controlling matrix as 
$$
C=\left(\begin{array}{cc}
I_p & 0 \\
0 & 0
\end{array}\right) ,
$$
then the Kalman type condition as
\begin{equation}\label{41}
    \operatorname{Rank}\left[C, B_0 C, \ldots, B_0^{k-1} C\right]=n,
\end{equation}
where $k$ is a positive integer.

The following theorem can be obtained.
\begin{thm} \label{thm:main}
	Under Assumption \eqref{41}, if $T > kL$, there exists a small $\varepsilon _0$ and a set of $C^1$ smooth control functions $H_i(t)$, $i=1,\cdots ,p$, such that for any $\varepsilon < \varepsilon _0 $ and any $\varPhi (x),\ \varphi  (x)$ which are $C^2$ smooth and satisfy the $C^2$ compatibility condition and $\varPsi (x),\ \psi (x)$ which are $C^1$ smooth and satisfy the $C^1$ compatibility condition, there exists a unique $C^2$ smooth solution $U(t,x)$ to the control problem \eqref{42}.
\end{thm}

The proof of Theorem \ref{thm:main} requires the following proposition.
\begin{prop}\label{prop:main}
	Taking $U^{(0)}\equiv 0$, consider the following iterative problem
	\begin{equation}
		\begin{cases}
		& \partial_t \widetilde{V}^{ \pm(m+1)} \pm a^{(m)} \partial_x \widetilde{V}^{ \pm(m+1)}=\mp \partial_x \widetilde{U}^{(m)}\left(\partial_t \pm a^{(m)} \partial_x\right)\left(a^{(m)}\right), \\
		& x=L: \widetilde{V}^{-(m+1)}=-\widetilde{V}^{+(m+1)}, \quad t \in \mathbb{R}^{+}, \\
		& x=0: \widetilde{V}^{-(m+1)}=B_0 \widetilde{V}^{+(m+1)}+\widetilde{H}^{(m+1)}(t) \\
		& +\left(\widetilde{D}^{+}\left(\widetilde{U}^{(m)}, \widetilde{U}_t^{(m)}, \widetilde{U}_x^{(m)}\right)-\widetilde{D}^{+}(0,0,0)\right) \widetilde{D}^{+}(0,0,0)^{-1} \widetilde{V}^{-(m)} \\
		& -\left(\widetilde{D}^{-}\left(\widetilde{U}^{(m)}, \widetilde{U}_t^{(m)}, \widetilde{U}_x^{(m)}\right)-\widetilde{D}^{-}(0,0,0)\right) \widetilde{D}^{+}(0,0,0)^{-1} \widetilde{V}^{+(m)}.\\
		& t=0:\left( U^{(m+1)},U^{(m+1)}_t \right) =\left( \varphi  \left( x \right) ,\psi \left( x \right) \right)  ,\\
		& t=T:\left( U^{(m+1)},U^{(m+1)}_t \right) =\left( \varPhi   \left( x \right) ,\varPsi  \left( x \right) \right) .
	\end{cases}	
	\end{equation}
where
$$
a^{(m)}=a\left(\widetilde{U}^{(m)}, \widetilde{U}_t^{(m)}, \widetilde{U}_x^{(m)}\right) .
$$
One can get uniform bounds
$$
 \left\lVert U^{\left( m \right)}\right\rVert  _{C^2}\leq C\varepsilon ,
$$
$$
 \left\lVert H^{\left( m \right)}\right\rVert  _{C^1}\leq C\varepsilon .
$$
as well as Cauchy sequence 
$$
\left\lVert  U^{\left( m \right)}-U^{\left( m-1 \right)}\right\rVert  _{C^1}\leq C\varepsilon \alpha ^{m-1},
$$
$$
\left\lVert  H^{\left( m \right)}-H^{\left( m-1 \right)}\right\rVert  _{C^0}\leq C\varepsilon \alpha ^{m-1}.
$$
with additional equi-continuity by estimates on modulus of continuity
$$
\omega \left( \eta \left| \frac{\partial H^{\left( m \right)}}{\partial t} \right. \right) +\omega \left(  \eta \left|\frac{\partial ^2U^{\left( m \right)}}{\partial t^2} \right. \right) +\omega \left( \eta \left|\frac{\partial ^2U^{\left( m \right)}}{\partial x^2} \right.  \right) +\omega \left( \eta \left|\frac{\partial ^2U^{\left( m \right)}}{\partial t\partial x} \right. \right) \leq \Omega \left( \eta \right) .
$$
\end{prop}
Where  $\Omega(\eta)$ is a monotonically increasing continuous function satisfying $\Omega(0)=0$.
Theorem \ref{thm:main} follows after obtaining Proposition \ref{prop:main}.In fact, The sequence $\{U ^{(m)}\} _{m=1} ^ \infty $ is a Cauchy sequence in $C^1$ space, and thus converges to some $C^1$ function $ U$ uniformly, applying Arzelà-Ascoli theorem, we know that there exists a subsequence of $\{U ^{(m)}\} _{m=1} ^\infty $ ,which converges uniformly in $C^2$ space. Thus we know that the whole original sequence$\{U ^{(m)}\} _{m=1} ^\infty $ converges to $U$ in $C^2$ space. Therefore, $U$ is $C^2$ smooth and satisfies the control condition, where $U^(m+1)$ for each step can be 
obtained using the Dirichlet boundary condition as
$$
U^{(m+1)}(t,x)=-\int_x^L \partial _x U^{(m+1)} \text{d} x,
$$
where
$$
\partial _x U^{(m+1)}=\frac{V^{-\left( m+1 \right) }-V^{+\left( m+1 \right)}}{ 2a\left({U}^{\left( m \right)} ,U_{t}^{\left( m \right)},U_{x}^{\left( m \right)} \right)}.
$$

\section{Proof of Theorem \ref{lem:linear}}
%%%%%%%%%%%%%%%%%%%%%%%%%%%%%%%%%%%%%%%%%%%%%%%%%%%%%%%%%%%%%%%%%%%%%%%%%%%%%%%%%%%%%%%%%%%%%%%%%%%%%%
We need to introduce a lemma about the linear equation.Consider the following linear equation system
\begin{equation}\label{e1}
	\left\{\begin{array}{l}
	\partial_t u_i- a(t,x)\partial_x u_i=f^{(1)}_i(t,x)(\partial_t - a(t,x)\partial_x)f^{(2)}_i(t,x), i=1, \cdots, r, \\
	\partial_t u_i+ a(t,x)\partial_x u_i=f^{(1)}_i(t,x)(\partial_t + a(t,x)\partial_x)f^{(2)}_i(t,x), i=r+1, \cdots, r+s.
	\end{array}\right.
\end{equation}
where $a(t,x)$ is a $C^1$ function satisfying that $\left\lVert a(t,x)-1\right\rVert _{C^1} \leq \varepsilon _0$.$\varepsilon _0$ is small enough. $f^{(\kappa )}_i(t,x),i=1,\cdots,r+s,\kappa =1,2$ are $C^1$ functions. The boundary conditions are given as follows:
\begin{equation}\label{e2}
	\left\{ \begin{array}{l}
		x=0:\ u_i=\sum_{j=1}^r{a_{ij}u_j + g_i(t) + h_i\left( t \right) ,\ i=r+1,\cdots ,n,}\\
		x=L:\ u_i=\sum_{j=r+1}^n{a_{ij}u_j +g_i(t) + h_i\left( t \right) ,\ i=1,\cdots ,r.}\\
	\end{array} \right. 
\end{equation}
where $g_i(t)$ is a $C^1$ function, $h_i(t)$ is a $C^1$ controlling function. The matrix $A=\left( a_{ij} \right) ,\ i=1,\cdots ,n,\ j=1,\cdots ,n$ is a $C^1$ matrix. $u(t,x)$ satisfies
\begin{equation}\label{4}
	u\left( 0,x \right) =\varphi \left( x \right) .
\end{equation}
The matrix $B$ is defined as follows:
$$
B=\left( \begin{matrix}
	I_{r_1}&		&		&		\\
	&		O_{r_2}&		&		\\
	&		&		I_{s_1}&		\\
	&		&		&		O_{s_2}\\
\end{matrix} \right) ,
$$
where $r_1 +r_2 =r,\ s_1+s_2=s, r+s=n $. The Kalman rank condition is given as follows:
\begin{equation}\label{kalman}
	\operatorname{Rank}\left[B, AB, \ldots, A^{k-1} B\right]=n,
\end{equation}
where $k$ is a positive integer. The following lemma is the main result of this section.
\begin{lem}\label{lem:linear}
Under the conditions of \eqref{kalman}, if $T>kL$, then there exists a set of $C^1$ smooth control functions $h_i(t)$, $i=1,\cdots ,r_1, r+1,\cdots ,r+s_1$, such that for any $C^1$ smooth functions $g_i(t), f_i(t,x)$, $i=1,\cdots ,n$ and any $C^1$ smooth functions $\varphi (x), \psi (x)$, which satisfy the $C^1$ compatibility conditions, there exists a unique $C^1$ smooth solution $u(t,x)$ to the control problem \eqref{e1}-\eqref{4} satisfying
	\begin{equation}\label{5}
        u(T,x)=\psi(x).
    \end{equation}
\end{lem}
\begin{proof}
	Without loss of generality, we can assume that $k$ is odd, $T-kL < \frac{L}{4}$ and $\varphi (x)=0$ ,or take any set of $\tilde{h}_i(t)$ that satisfies the compatibility conditions at $(0,0)$ and $(0,L)$ and consider the equations of $\tilde{u}(t,x)$ satisfying $\tilde{u}(0,x)= \varphi(x)$ below:
	\begin{equation}\label{e4}
		\left\{\begin{array}{l}
		\partial_t \tilde{u} _i- a(t,x)\partial_x \tilde{u} _i=\chi (t)f^{(1)}_i(t,x)(\partial_t - a(t,x)\partial_x)f^{(2)}_i(t,x), i=1, \cdots, r, \\
		\partial_t \tilde{u} _i+ a(t,x)\partial_x \tilde{u} _i=\chi (t)f^{(1)}_i(t,x)(\partial_t + a(t,x)\partial_x)f^{(2)}_i(t,x), i=r+1, \cdots, r+s.
		\end{array}\right.
	\end{equation}
where $\chi (t)$ is a smooth function that satisfies $\chi (t)=1$ for $t\in [0,\frac{L}{2}]$ and $\chi (t)=0$ for $t\in [T-\frac{L}{2},T]$. The boundary conditions are given as follows:
\begin{equation}\label{e5}
		\left\{ \begin{array}{l}
		x=0:\ \tilde{u} _i=\sum_{j=1}^r{a_{ij}\tilde{u} _j + \chi (t)g_i(t) + \tilde{h} _i\left( t \right) ,\ i=r+1,\cdots ,n,}\\
		x=L:\ \tilde{u} _i=\sum_{j=r+1}^n{a_{ij}\tilde{u} _j +\chi (t)g_i(t) + \tilde{h} _i\left( t \right) ,\ i=1,\cdots ,r.}
		\end{array} \right. 
	\end{equation}
Then we can get that $\tilde{u}(T,x)$ satisfies the $C^1$ compatibility conditions at $(T,0)$ and $(T,L)$ and consider the following controllability problem:
	\begin{equation}\label{e6}
		\left\{\begin{array}{l}
		\partial_t u_i- a(t,x)\partial_x u_i=(1-\chi (t))f^{(1)}_i(t,x)(\partial_t - a(t,x)\partial_x)f^{(2)}_i(t,x), i=1, \cdots, r, \\
		\partial_t u_i+ a(t,x)\partial_x u_i=(1-\chi (t))f^{(1)}_i(t,x)(\partial_t + a(t,x)\partial_x)f^{(2)}_i(t,x), i=r+1, \cdots, r+s.
		\end{array}\right.
	\end{equation}
where the boundary conditions are given as follows:
\begin{equation}\label{e7}
		\left\{ \begin{array}{l}
		x=0:\ u_i=\sum_{j=1}^r{a_{ij}u_j + (1-\chi (t))g_i(t) + h_i\left( t \right) ,\ i=r+1,\cdots ,n,}\\
		x=L:\ u_i=\sum_{j=r+1}^n{a_{ij}u_j +(1-\chi (t))g_i(t) + h_i\left( t \right) ,\ i=1,\cdots ,r.}\\
		\end{array} \right. 
	\end{equation}
The initial and final conditions are given as follows:
\begin{equation}\label{e8}
		u(0,x)=0 ,\quad u(T,x)=\psi (x)- \tilde{u}(T,x).
	\end{equation}
Finally, the two sets of control functions and solutions obtained can be added separately.
We determine $h_i\left( t \right)$ in the following order:
In the following we first study the equations of characteristic lines. Let 
$$
b(t,x)=\frac{1}{a(t,x)}.
$$
The characteristic lines of the system \eqref{e1} are given as follows:
\begin{equation}\label{27}
    \left\{ \begin{array}{l}
        \frac{\text{d}\tilde{t} _p}{\text{d}x}= -b\left( \tilde{t} _p ,x \right), \\
        \tilde{t} _p(0;x_0) = t_{p-1}(0;x_0) ,\ p=2,\cdots ,k+1,\\
        \tilde{t} _1(x_0;x_0) = T,
    \end{array} \right. 
    \end{equation}
\begin{equation}\label{28}
    \left\{ \begin{array}{l}
        \frac{\text{d}t_p}{\text{d}x}= b\left( t_p,x \right) ,\\
        t_p(L;x_0) = \tilde{t} _{p-1}(L;x_0) ,\ p=2,\cdots ,k+1,\\
        t_1(x_0;x_0) = T,
    \end{array} \right. 
\end{equation}
thereby obtaining the following equations:
$$
\begin{aligned}
	t_1 \left( x;x_0 \right) &=T+\int_{x_0}^x b\left( t_1 \left( s;x_0 \right) ,s \right) \text{d}s ,\\
    t_p \left( x;x_0 \right) &=t_{p-1} \left( L;x_0 \right)+\int_{L}^x b\left( t_p \left( s;x_0 \right) ,s \right) \text{d}s , p=2 , \cdots , k+1 ,\\
	\partial _{x_0}t_1\left( x;x_0 \right) &=-b\left( T,x_0 \right) +\int_{x_0}^x{ \partial _t}b\left( t_1 \left( s;x_0 \right) ,s \right) \partial _{x_0}t_1\left( s;x_0 \right) \text{d}s ,\\
	\partial _{x_0}t_p\left( x;x_0 \right) &=\partial _{x_0} t _{p-1} \left( L;x_0 \right) +\int_{L}^x{ \partial _t}b\left( t_p \left( s;x_0 \right) ,s \right) \partial _{x_0}t_p \left( s;x_0 \right) \text{d}s , p=2 , \cdots , k+1 .
\end{aligned}
$$
Denote $z_p(x;x_0)=\partial _{x_0}t_p \left( x;x_0 \right)$, which conduces that
$$
\frac{\text{d}}{\text{d}x}z_p \left( x;x_0 \right) = \partial _t b_p \left( t\left( x;x_0 \right) ,x \right) z_p \left( x;x_0 \right) .
$$
Then we can get the following equations:
\begin{equation}\label{time line}
	\begin{aligned}
	z_1 \left( x;x_0 \right) &=-b\left( T,x_0 \right) \exp \left(  \int_{x_0}^x{\partial _t}b\left( t_1 \left( s;x_0 \right) ,s \right) \text{d}s \right), \\
	z_p \left( x;x_0 \right) &=\partial _{x_0} t_{p-1} \left( L;x_0 \right) \exp \left(  \int_{L}^x{\partial _t}b\left( t_p \left( s;x_0 \right) ,s \right) \text{d}s \right).
\end{aligned}
\end{equation}

Similarly one can obtain that
$$
\begin{aligned}
	\tilde{t}  _1 \left( x;x_0 \right) &=T-\int_{x_0}^x b\left( \tilde{t}  _1 \left( s;x_0 \right) ,s \right) \text{d}s\\
    \tilde{t} _p \left( x;x_0 \right) &=t _{p-1} \left( 0;x_0 \right)-\int_{0}^x b\left( \tilde{t} _p \left( s;x_0 \right) ,s \right) \text{d}s , l=2 , \cdots , k+1 ,\\
	\tilde{z} _1 \left( x;x_0 \right) &=b\left( T,x_0 \right) \exp \left(  -\int_{x_0}^x{\partial _t}b\left(\tilde{ t} _1 \left( s;x_0 \right) ,s \right) \text{d}s \right), \\
	\tilde{z} _p \left( x;x_0 \right) &=\partial _{x_0} t _{p-1} \left( 0;x_0 \right) \exp \left(  \int_{0}^x{\partial _t}b\left( \tilde{t} _p \left( s;x_0 \right) ,s \right) \text{d}s \right).\\
\end{aligned}
$$
Therefore
\begin{equation}\label{29}
    \left\{ \begin{array}{l}
		t_{1}(x;0) =T,\\
        t_{2m+1}(x;0) = t_{2m}(x;0) ,\\
        t_{2m}(x;L) = t_{2m - 1}(x;L) ,
    \end{array} \right. 
    \end{equation}
\begin{equation}\label{210}
    \left\{ \begin{array}{l}
		\tilde{t} _{1}(x;L) =T,\\
        \tilde{t} _{2m+1}(x;L) = \tilde{t} _{2m}(x;L) ,\\
        \tilde{t} _{2m}(x;0) = \tilde{t} _{2m - 1}(x;0) .
    \end{array} \right. 
    \end{equation}
and
\begin{equation}\label{211}
    \left\{ \begin{array}{l}
		\partial _{x_0} t_{1}(x;0) = -b(T,0),\\
        \partial _{x_0} t_{2m+1}(x;0) = -\partial _{x_0} t_{2m}(x;0) ,\\
        \partial _{x_0} t_{2m}(x;L) = -\partial _{x_0} t_{2m - 1}(x;L) ,
    \end{array} \right. 
    \end{equation}
\begin{equation}\label{212}
    \left\{ \begin{array}{l}
		\partial _{x_0} \tilde{t} _{1}(x;L) = b(T,L),\\
        \partial _{x_0}\tilde{t} _{2m+1}(x;L) =- \partial _{x_0}\tilde{t}_{2m}(x;L) ,\\
        \partial _{x_0}\tilde{t}_{2m}(x;0) = -\partial _{x_0}\tilde{t}_{2m - 1}(x;0) .
    \end{array} \right. 
\end{equation}
The following defines a set of vector functions:
$$
H_p\left( x \right) =\left( H^{-}\left( \tilde{t}  _{p}(L;x) \right) ,H^{+}\left( t_{p}(0;x)\right) \right) ^T,
$$
where
$$
H^{-}\left( t \right) =\left( h_1\left( t \right) ,\cdots ,h_{r_1}\left( t \right) ,0,\cdots,0 \right) ,
$$
$$
H^{+}\left( t \right) =\left( h_{r+1}\left( t \right) ,\cdots ,h_{r+s_1}\left( t \right) ,0,\cdots,0 \right) .
$$
Let
$$
h_{i}\left( x\right) =0,\,\,x < 0 ,i = 1, \dots , n.
$$
Define
$$
u\left(t, x \right) =\left( u^-\left( t,x \right) ,u^{+}\left( t,x\right)   \right) ^T,
$$
where
$$
u^{-}\left( t,x \right) =\left( u_1\left( t ,x\right) ,\cdots ,u_{r}\left( t,x \right) \right) ,
$$
$$
u^{+}\left( t,x \right) =\left( u_{r+1}\left( t ,x\right) ,\cdots ,u_{r+s}\left( t ,x\right) \right) .
$$
Define
$$
G_p\left( x \right) =\left( G^-\left( \tilde{t}  _{p}(L;x) \right) ,G^{+}\left( t_{p}(0;x)\right)   \right) ^T,
$$
where
$$
G^{-}\left( t \right) =\left( g_1\left( t \right) ,\cdots ,g_{r}\left( t \right) \right) ,
$$
$$
G^{+}\left( t \right) =\left( g_{r+1}\left( t \right) ,\cdots ,g_{r+s}\left( t \right) \right) .
$$
Define
$$
F_1\left( x \right) =\left(\int_{x}^L{F} ^{-}\left( \tilde{t} _1\left( s;x \right) ,s \right) \text{d}s ,\int_{0}^{x} {F}^{+}\left( t_1\left( s;x \right) ,s \right) \text{d}s  \right) ^T,
$$
$$
F_p\left( x \right) =\left( \int_0^L{F} ^{-}\left( \tilde{t} _p\left( s;x \right) ,s \right) \text{d}s ,\int_0^L{F} ^{+}\left( t_p\left( s;x \right) ,s \right) \text{d}s  \right) ^T,p=2,\cdots,k+1,
$$
where
$$
F^{-}\left( t ,x\right) =\left( f^{(1)}_1(t,x)(b(t,x)\partial_t - \partial_x)f^{(2)}_1(t,x) ,\cdots ,f^{(1)}_r(t,x)(b(t,x)\partial_t - \partial_x)f^{(2)}_r(t,x)\right) ,
$$
$$
F^{+}\left( t ,x\right) =\left( f^{(1)}_{r+1}(t,x)(b(t,x)\partial_t - \partial_x)f^{(2)}_1(t,x) ,\cdots ,f^{(1)}_{r+s}(t,x)(b(t,x)\partial_t - \partial_x)f^{(2)}_{r+s}(t,x)\right) ,
$$
If a solution $u(t,x)$ satisfying the requirement exists,for any $x\in [0,L]$, we can get the following equations with characteristic lines method:
\begin{equation}\label{F}
	\begin{aligned}
	\psi\left( x \right) &=u(T,x) \\
							   &=(u^- (\tilde{t} _1(L;x) ,L) ,u^+ (t _1(0;x),0))^T + F_1(x)\\
							   &=A(u^- (t _1(0;x) ,0) ,u^+ (\tilde{t} _2 (L;x),L))^T + BH _1(x)+G_1(x)+F_1(x)\\
							   &=A(u^- (t _2 ,L) ,u^+ (\tilde{t} _2 ,0))^T +AF_2(x)+ BH _1(x)+G_1(x)+F_1(x)\\
							   & \cdots \\
							   &=A^kB(H_{k+1}\left( x \right)+G_{k+1}\left( x \right)+F_{k+1}\left( x \right)) +\cdots  +B(H_1\left( x \right)+G_1\left( x \right)+F_1\left( x \right)) ,\ x\in \left[ 0,L \right].
	\end{aligned}
	\end{equation}
Using this equation, we first determine the value of the function and derivative of $h_i(t_p(0;0)),h_i(t_p(0;L)),h_i(t_p(L;0)),h_i(t_p(L;L))$,$(i=1,\cdots,n; p=1,\cdots,k+1)$.
Consider the vector $\tilde{G}  (x) = \sum_{l=0}^k A^l G _{l+1}(x)$.
$$
\tilde{G}^-  (L) =G^- _1 (L) +A_1 G^+ _2 (L) +A_1 A_2 G^- _3 (L)  +\cdots +A_1 A_2 \cdots A_1 G^+ _{k+1} (L) ,
$$
$$
\tilde{G}^+  (L) =G^+ _1 (L) +A_2 G^- _2 (L) +A_2 A_1 G^+ _3 (L)  +\cdots +A_2 A_1 \cdots A_2 G^- _{k+1} (L) .
$$
Recognizing \eqref{29}-\eqref{210} and $T-kL < \frac{L}{4}$, we can get 
$$
G^- _{k+1}(L)=0,
$$
$$
G^+ _{m}(L) =G^+ _{m+1}(L) ,\ m is odd ,\ m=1,\cdots ,k,
$$
$$
G^+ _{m}(L) =G^+ _{m+1}(L) ,\ m is even ,\ m=1,\cdots ,k.
$$
Therefore,
\begin{equation}
	\begin{aligned}
		\tilde{G}^+  (L) &=G^+ _1 (L) +A_2 G^- _2 (L) +A_2 A_1 G^+ _3 (L)  +\cdots +A_2 A_1 \cdots A_1 G^+ _{k} (L) \\
		                 &=G^+ _2 (L) +A_2 G^- _3 (L) +A_2 A_1 G^+ _4 (L)  +\cdots +A_2 A_1 \cdots A_1 G^+ _{k+1} (L) ,
	\end{aligned}
\end{equation}
and thereby we can get the following equations:
$$
\tilde{G}^-  (L) =G^- _1 (L) +A_1 \tilde{G} ^+  (L) .
$$
Similarly,
$$
\tilde{G}^+  (0) =G^+ _1 (0) +A_2 \tilde{G} ^-  (0) .
$$
Namely, 
\begin{equation}\label{20}
    \binom{\tilde{G} ^{-}(L)}{\tilde{G}^{+}(0)}=\left(\begin{array}{ll} 
        & A_1 \\
        A_2 &
    \end{array}\right)\binom{\tilde{G}^{-}(0)}{\tilde{G} ^{+}(L)}+ \binom{g ^{-}(T)}{g ^{+}(T)}.
\end{equation}
Now consider the vector $\tilde{F}  (x) = \sum_{l=0}^k A^l F _{l+1}(x)$.
$$
\tilde{F}^-  (L) =F^- _1 (L) +A_1 F^+ _2 (L) +A_1 A_2 F^- _3 (L)  +\cdots +A_1 A_2 \cdots A_1 F^+ _{k+1} (L) ,
$$
$$
\tilde{F}^+  (L) =F^+ _1 (L) +A_2 F^- _2 (L) +A_2 A_1 F^+ _3 (L)  +\cdots +A_2 A_1 \cdots A_2 F^- _{k+1} (L) .
$$
From the defination of $\chi (t)$ ,we can assume that $f^{(j)}_i(t,x)=0,t < T-\frac{L}{4},j=1,2,i=1,\cdots,n$.Recognizing \eqref{29}-\eqref{210} and $T-kL < \frac{L}{4}$, we can get 
$$
F^- _{1}(L)=0,
$$
$$
F^- _{k+1}(L)=0,
$$
$$
F^+ _{m}(L) =F^+ _{m+1}(L) ,\ m is odd ,\ m=1,\cdots ,k,
$$
$$
F^+ _{m}(L) =F^+ _{m+1}(L) ,\ m is even ,\ m=1,\cdots ,k.
$$
Therefore,
\begin{equation}
	\begin{aligned}
		\tilde{F}^+  (L) &=F^+ _1 (L) +A_2 F^- _2 (L) +A_2 A_1 F^+ _3 (L)  +\cdots +A_2 A_1 \cdots A_1 F^+ _{k} (L) \\
		                 &=F^+ _2 (L) +A_2 F^- _3 (L) +A_2 A_1 F^+ _4 (L)  +\cdots +A_2 A_1 \cdots A_1 F^+ _{k+1} (L) ,
	\end{aligned}
\end{equation}
and thereby we can get the following equations:
$$
\tilde{F}^-  (L) =A_1 \tilde{F} ^+  (L) .
$$
Similarly,
$$
\tilde{F}^+  (0) =A_2 \tilde{F} ^-  (0) .
$$
Namely, 
\begin{equation}\label{compatibility condition of F}
    \binom{\tilde{F} ^{-}(L)}{\tilde{F}^{+}(0)}=\left(\begin{array}{ll} 
        & A_1 \\
        A_2 &
    \end{array}\right)\binom{\tilde{F}^{-}(0)}{\tilde{F} ^{+}(L)}.
\end{equation}
The $C^0$ compatibility condition satisfied by $\psi(x)$ is
\begin{equation}\label{21}
    \psi _i\left( 0 \right) =\sum_{j=1}^r{a_{ij}\psi _j\left( 0 \right) + g_i(T),\,\,i=r+s_1+1,\cdots ,n},
\end{equation}
\begin{equation}\label{22}
    \psi _i\left( L \right) =\sum_{j=r+1}^n{a_{ij}\psi _j\left( L \right) +g_i(T),\,\,i=r_1+1,\cdots ,r}.
\end{equation}
So $\psi -\tilde{G} -\tilde{F} $ satisfies the conditions of Lemma \ref{l50}. Therefore, the following equations have a solution:
\begin{equation}\label{c0 continuity at 0}
\psi\left( 0 \right) =A^k\left( H_{k+1}\left( 0 \right) +G_{k+1}\left( 0 \right)+F_{k+1}\left( 0 \right) \right)  +\cdots +\left( H_1\left( 0 \right) +G_1\left( 0 \right)+F_1\left( 0 \right) \right) ,
\end{equation}
\begin{equation}\label{c0 continuity at L}
\psi\left(L \right) =A^k\left( H_{k+1}\left( L \right) +G_{k+1}\left( L \right)+F_{k+1}\left( L \right) \right)+\cdots +\left( H_1\left( L \right) +G_1\left( L \right)+F_1\left( L \right) \right) ,
\end{equation}
to determine the values of $h_i$ at $t_p(0;0),t_p(0;L),t_p(L;0),t_p(L;L)$,$(i=1,\cdots,n; p=1,\cdots,k+1)$.
Derivation of  \eqref{F} with respect to $x$ gives the following equations:
\begin{equation}\label{derivative of psi}
	\psi  '\left( x \right)=A^kB(H'_{k+1}\left( x \right)+G'_{k+1}\left( x \right)+F'_{k+1}\left( x \right)) +\cdots + B(H'_1\left( x \right)+G'_1\left( x \right)+F'_1\left( x \right)) ,\ x\in \left[ 0,L \right],
\end{equation}
where 
$$
H'_{p}\left( x \right) =\left( \frac{\text{d} H^{-}}{ \text{d}t} \left( \tilde{t}_{p}(L;x) \right) \frac{\partial \tilde{t} _{p}(L;x)}{\partial x} ,\frac{\text{d} H^{+}}{\text{d} t}\left( t_{p}(0;x)\right) \frac{\partial t_{p}(0;x)}{\partial x} \right) ^T,
$$
$$
G'_{p}\left( x \right) =\left( \frac{\text{d} G^{-}}{\text{d} t}\left( \tilde{t}_{p}(L;x) \right) \frac{\partial \tilde{t}  _{p}(L;x)}{\partial x} ,\frac{\text{d} G^{+}}{\text{d} t}\left( t_{p}(0;x)\right) \frac{\partial t_{p}(0;x)}{\partial x} \right) ^T,
$$
$$
F_{1}^{'}\left( x \right) =\left( \begin{matrix}
	-f^{\left( 1 \right) -}\left( b\partial _t-\partial _x \right) f^{\left( 2 \right) -}\left( T,x \right) +\int_x^L{\frac{\partial f^{\left( 1 \right) -}}{\partial t}}\left( b\partial _t-\partial _x \right) f^{\left( 2 \right) -}\left( \tilde{t}_1\left( s,x \right) ,s \right) \frac{\partial \tilde{t}_1}{\partial x}\left( s;x \right) ds-\int_x^L{\frac{\partial f^{\left( 1 \right) -}}{\partial t}}\left( b\partial _t-\partial _x \right) f^{\left( 2 \right) -}\left( \tilde{t}_1\left( s,x \right) ,s \right) \frac{\partial \tilde{t}_1}{\partial x}\left( s;x \right) ds+\left. f^{\left( 1 \right) -}\frac{\partial f^{\left( 2 \right) -}}{\partial t}\left( \tilde{t}_1\left( s;x \right) ,s \right) \frac{\partial \tilde{t}_1}{\partial x}\left( s;x \right) \right|_{x}^{L}&		f^{\left( 1 \right) +}\left( b\partial _t-\partial _x \right) f^{\left( 2 \right) +}\left( T,x \right) +\int_0^x{\frac{\partial f^{\left( 1 \right) +}}{\partial t}}\left( b\partial _t-\partial _x \right) f^{\left( 2 \right) +}\left( t_1\left( s,x \right) ,s \right) \frac{\partial t_1}{\partial x}\left( s;x \right) ds-\int_0^x{\frac{\partial f^{\left( 1 \right) +}}{\partial t}}\left( b\partial _t-\partial _x \right) f^{\left( 2 \right) +}\left( t_1\left( s,x \right) ,s \right) \frac{\partial t_1}{\partial x}\left( s;x \right) ds+\left. f^{\left( 1 \right) +}\frac{\partial f^{\left( 2 \right) +}}{\partial t}\left( t_1\left( s;x \right) ,s \right) \frac{\partial t_1}{\partial x}\left( s;x \right) \right|_{0}^{x}\\
\end{matrix} \right) 
$$
$$
F_{p}^{'}\left( x \right) =\left( \begin{matrix}
	\int_0^L{\frac{\partial f^{\left( 1 \right) -}}{\partial t}}\left( b\partial _t-\partial _x \right) f^{\left( 2 \right) -}\left( \tilde{t}_p\left( s,x \right) ,s \right) \frac{\partial \tilde{t}_p}{\partial x}\left( s;x \right) ds-\int_0^L{\frac{\partial f^{\left( 1 \right) -}}{\partial t}}\left( b\partial _t-\partial _x \right) f^{\left( 2 \right) -}\left( \tilde{t}_p\left( s,x \right) ,s \right) \frac{\partial \tilde{t}_p}{\partial x}\left( s;x \right) ds+\left. f^{\left( 1 \right) -}\frac{\partial f^{\left( 2 \right) -}}{\partial t}\left( \tilde{t}_p\left( s;x \right) ,s \right) \frac{\partial \tilde{t}_p}{\partial x}\left( s;x \right) \right|_{0}^{L}&		\int_0^L{\frac{\partial f^{\left( 1 \right) +}}{\partial t}}\left( b\partial _t-\partial _x \right) f^{\left( 2 \right) +}\left( t_p\left( s,x \right) ,s \right) \frac{\partial t_p}{\partial x}\left( s;x \right) ds-\int_0^L{\frac{\partial f^{\left( 1 \right) +}}{\partial t}}\left( b\partial _t-\partial _x \right) f^{\left( 2 \right) +}\left( t_p\left( s,x \right) ,s \right) \frac{\partial t_p}{\partial x}\left( s;x \right) ds+\left. f^{\left( 1 \right) +}\frac{\partial f^{\left( 2 \right) +}}{\partial t}\left( t_p\left( s;x \right) ,s \right) \frac{\partial t_p}{\partial x}\left( s;x \right) \right|_{0}^{L}\\
\end{matrix} \right) 
$$
Consider the vector $\tilde{G'}  (x) = \sum_{l=0}^k A^l G '_{l+1}(x)$ and  $\tilde{F'}  (x) = \sum_{l=0}^k A^l F '_{l+1}(x)$.Similar to the previous case using \eqref{29}-\eqref{212}, we can get the following equations:
$$
\tilde{G}'_-\left( L \right) =b\left( T,L \right) g'_r\left( T \right) -A_1\tilde{G'}_+\left( L \right) 
$$
$$
\tilde{G}'_+\left( 0 \right) =-b\left( T,L \right) g'_s\left( T \right) -A_2\tilde{G'}_-\left( 0 \right) 
$$
$$
\tilde{F}'_-\left( L \right) +f_{r}^{\left( 1 \right)}\left( T,L \right) \partial _sf_{r}^{\left( 2 \right)}\left( T,L \right) =A_1\left( -\tilde{F}'_+\left( L \right) +f_{s}^{\left( 1 \right)}\left( T,L \right) \partial _sf_{s}^{\left( 2 \right)}\left( T,L \right) \right) 
$$
$$
-\tilde{F}'_+\left( 0 \right) +f_{s}^{\left( 1 \right)}\left( T,0 \right) \partial _sf_{s}^{\left( 2 \right)}\left( T,0 \right) =A_2\left( \tilde{F}'_-\left( L \right) +f_{r}^{\left( 1 \right)}\left( T,0 \right) \partial _sf_{r}^{\left( 2 \right)}\left( T,0 \right) \right) 
$$
Combined with the $C^1$ compatibility condition satisfied by $\psi (x)$:
$$
a\left( T,L \right) \psi '_r\left( L \right) +f_{r}^{\left( 1 \right)}\left( \partial _t-a\left( T,L \right) \partial _x \right) f_{r}^{\left( 2 \right)}\left( T,L \right) =A_2\left( -a\left( T,L \right) \psi '_s\left( L \right) +f_{s}^{\left( 1 \right)}\left( \partial _t+a\left( T,L \right) \partial _x \right) f_{s}^{\left( 2 \right)}\left( T,L \right) \right) +g'_r\left( T \right) +h'_r\left( T \right) 
$$
$$
-a\left( T,0 \right) \psi '_s\left( 0 \right) +f_{s}^{\left( 1 \right)}\left( \partial _t+a\left( T,0 \right) \partial _x \right) f_{s}^{\left( 2 \right)}\left( T,0 \right) =A_2\left( a\left( T,0 \right) \psi '_r\left( 0 \right) +f_{r}^{\left( 1 \right)}\left( \partial _t-a\left( T,0 \right) \partial _x \right) f_{r}^{\left( 2 \right)}\left( T,0 \right) \right) +g'_s\left( T \right) +h'_s\left( T \right) 
$$
we know that the vector $(\psi '_r-\tilde{G}'_--\tilde{F}'_-,-\psi '_s+\tilde{G}'_++\tilde{F}'_+)^T$ stastisfies the conditions of Lemma \ref{l50}.Therefore, the following equations have a solution:
\begin{equation}\label{c1 continuity at 0}
	\psi  '\left( 0 \right)=A^kB(H'_{k+1}\left( 0\right)+G'_{k+1}\left( 0 \right)+F'_{k+1}\left( 0 \right)) +\cdots + B(H'_1\left( 0 \right)+G'_1\left( 0 \right)+F'_1\left( 0 \right)) ,
\end{equation}
\begin{equation}\label{c1 continuity at L}
	\psi  '\left( L \right)=A^kB(H'_{k+1}\left( L \right)+G'_{k+1}\left( L \right)+F'_{k+1}\left( L \right)) +\cdots + B(H'_1\left( L \right)+G'_1\left( L \right)+F'_1\left( L \right)) ,
\end{equation}
to determine the values of $h'_i$ at $t_p(0;0),t_p(0;L),t_p(L;0),t_p(L;L)$,$(i=1,\cdots,n; p=1,\cdots,k+1)$.
Now we continue to determine the other values of $h_i(t)$.\eqref{F} can be written as
\begin{equation}\label{G}
    \psi\left( x \right) =A^kBH_{k+1}\left( x \right) +\left[B, AB, \ldots, A^{k-1} B\right] \hat{H}(x)  ,\ x\in \left( 0,L \right),
\end{equation}
where 
$$
\hat{H}(x)=\left( H^T_1\left( x \right) ,\cdots ,H^T_k\left( x \right) \right)^T .
$$
From \eqref{kalman} and the implicit function existence theorem, we know that there exist $n$ components of $\hat{H}(x)$ denoted as $H[n](x) $,s.t.
\begin{equation}\label{H}
H\left[ n \right] \left( x \right) =\mathcal{H} \left( \psi\left( x \right) ,H_{k+1}\left( x \right) ,\hat{H}\left[ \hat{n} \right] \left( x \right) \right) ,
\end{equation}
where $\hat{H}\left[ \hat{n} \right] \left( x \right)$ is the other components of $\hat{H}(x)$, and $\mathcal{H}$ is a $C^1$ continuous function. For $H_{k+1}\left( x \right) $, we utilize the interpolation method to obtain the following equations:
\begin{equation}\label{polynomial k+1 r}
h_r\left( t_{k+1}\left( L;x \right) \right) =p\left( L-x;h_r\left( t_{k+1}\left( L;0 \right) \right) ,-h'_r\left( t_{k+1}\left( L;0 \right) \right) \right) 
\end{equation}
\begin{equation}\label{polynomial k+1 s}
	h_s\left( t_{k+1}\left( 0;x \right) \right) =p\left( x;h_s\left( t_{k+1}\left( 0;L \right) \right) ,h'_s\left( t_{k+1}\left( 0;L \right) \right) \right) 
\end{equation}
where $p(x;\alpha ,\alpha  ') ,\ x \in [0,L]$ is a polynomial determined by $\alpha ,\alpha  '$ satisfying $p(x;\alpha ,\alpha  ') =0, \  x \in [0,\frac{7L}{8}]$ and $p(L)=\alpha ,\ p'(L)=\alpha  '$ .Similarly,for the components of $\hat{H}\left[ \hat{n} \right] \left( x \right)$, we can also use the interpolation method to obtain the following equations:
\begin{equation}\label{polynomial p r}
	h_r\left( \tilde{t}_p\left( L;x \right) \right) =p\left( x;h_r\left( \tilde{t}_p\left( L;0 \right) \right) ,h'_r\left( \tilde{t}_p\left( L;0 \right) \right) ,h_r\left( \tilde{t}_p\left( L;L \right) \right) ,h'_r\left( \tilde{t}_p\left( L;L \right) \right) \right) ,p=1,\cdots,k.
\end{equation}
\begin{equation}\label{polynomial p s}
	h_s\left( t_p\left( 0;x \right) \right) =p\left( x;h_s\left( t_p\left( 0;0 \right) \right) ,h'_s\left( t_p\left( 0;0 \right) \right) ,h_s\left( t_p\left( 0;L \right) \right) ,h'_s\left( t_p\left( 0;L \right) \right) \right) ,p=1,\cdots,k.
\end{equation}
where $p(x;\alpha ,\alpha  ',\beta ,\beta  ') ,\ x \in [0,L]$ is a polynomial determined by $\alpha ,\alpha  ',\beta ,\beta  '$ satisfying $p(0)=\alpha ,\ p'(0)=\alpha  ',\ p(L)=\beta ,\ p'(L)=\beta  '$ . Finally, we use \eqref{H} to obtain the values of $H\left[ n \right] \left( x \right),\  x \in (0,L)$.From the $C^1$ continuity of $\mathcal{H}$ , \eqref{c0 continuity at 0} and \eqref{c0 continuity at L}, \eqref{c1 continuity at 0} and \eqref{c1 continuity at L} and \eqref{kalman} ,we know that $h_i(t)$ is $C^1$ continuous in $[0,T]$ .
A solution $u(t,x)$ is obtained using the obtained boundary conditions and initial value conditions. Then
\begin{equation}
	\begin{aligned}
		u(T,x)				   &=(u^- (\tilde{t} _1(L;x) ,L) ,u^+ (t _1(0;x),0))^T + F_1(x)\\
							   &=A(u^- (t _1(0;x) ,0) ,u^+ (\tilde{t} _2 (L;x),L))^T + BH_1(x)+G_1(x)+F_1(x)\\
							   &=A(u^- (t _2 ,L) ,u^+ (\tilde{t} _2 ,0))^T +AF_2(x)+ BH_1(x)+G_1(x)+F_1(x)\\
							   & \cdots \\
							   &=A^kB(H_{k+1}\left( x \right)+G_{k+1}\left( x \right)+F_{k+1}\left( x \right)) +\cdots  +B(H_1\left( x \right)+G_1\left( x \right)+F_1\left( x \right)) ,\ x\in \left[ 0,L \right].
	\end{aligned}
\end{equation}
Therefore, $u(T,x)=\psi(x)$ and thereby $h_i(t),i=1,\cdots,n$ are exactly the control function that satisfies the requirement.
\end{proof}
From the proof above, it is not difficult to obtain the $C^1$ estimate for $u(t,x)$ and $h_i(t)$, $i=1,\cdots,n$.
From \eqref{time line}, we get
\begin{equation}\label{estimate of time line}
	\lVert z_p\left( x;x_0 \right) -1 \rVert _{c^0}\leq \lVert b-1 \rVert _{c^1}\left( 1+C\lVert \partial _tb \rVert _{c^0} \right) ^p
\end{equation}
Using \eqref{c0 continuity at 0}, \eqref{c0 continuity at L}, \eqref{c1 continuity at 0}, \eqref{c1 continuity at L} and \eqref{estimate of time line}, we know the estimate of $h_i(t)$, $i=1,\cdots,n$ at $t_p(0;0),t_p(0;L),t_p(L;0),t_p(L;L)$,$(p=1,\cdots,k+1)$.
\begin{equation}\label{estimate of h_i at t_p}
	\left\| h_i(t_p(\lambda _1;\lambda _2)) \right\|_{C^1} \leq C_{A,k,\lVert a-1 \rVert _{C^1}}(\lVert \psi \rVert _{C^1}+\lVert \varphi  \rVert _{C^1}+  \lVert g \rVert _{C^1}+\lVert f^{(1)} \rVert _{C^1}\lVert f^{(2)} \rVert _{C^1}) ,\lambda _1,\lambda _2 \in \{ 0,L \} ,\ p=1,\cdots,k+1.
\end{equation}
From the defination $H_{k+1}(x)$ and $\hat{H}\left[ \hat{n} \right] \left( x \right)$, we know that
\begin{equation}\label{estimate of H}
	\left\| H_{k+1}(x) \right\|_{C^1} \leq C_{A,k,\lVert a-1 \rVert _{C^1}}(\lVert \psi \rVert _{C^1}+\lVert \varphi  \rVert _{C^1}+  \lVert g \rVert _{C^1}+\lVert f^{(1)} \rVert _{C^1}\lVert f^{(2)} \rVert _{C^1}) ,\ x\in [0,L].
\end{equation}
and
\begin{equation}\label{estimate of hat{H}}
	\left\| \hat{H}\left[ \hat{n} \right] \left( x \right) \right\|_{C^1} \leq C_{A,k,\lVert a-1 \rVert _{C^1}}(\lVert \psi \rVert _{C^1}+\lVert \varphi  \rVert _{C^1}+  \lVert g \rVert _{C^1}+\lVert f^{(1)} \rVert _{C^1}\lVert f^{(2)} \rVert _{C^1}) ,\ x\in [0,L].
\end{equation}
Using \eqref{H}, we can get the estimate of $H\left[ n \right] \left( x \right)$:
\begin{equation}\label{estimate of H[n]}
	\left\| H\left[ n \right] \left( x \right) \right\|_{C^1} \leq C_{A,k,\lVert a-1 \rVert _{C^1}}(\lVert \psi \rVert _{C^1}+\lVert \varphi  \rVert _{C^1}+  \lVert g \rVert _{C^1}+\lVert f^{(1)} \rVert _{C^1}\lVert f^{(2)} \rVert _{C^1}) ,\ x\in [0,L].
\end{equation}
Namely, $\lVert h_i(t) \rVert _{C^1}  \leq C_{A,k,\lVert a-1 \rVert _{C^1}}(\lVert \psi \rVert _{C^1}+\lVert \varphi  \rVert _{C^1}+  \lVert g \rVert _{C^1}+\lVert f^{(1)} \rVert _{C^1}\lVert f^{(2)} \rVert _{C^1}) $,$ t \in [0,T]$,  $i=1,\cdots,n$ .
Then we try to get the estimate of $u(t,x)$:
From \eqref{e1}, we know that
$$
\frac{\mathrm{d}u_s}{\mathrm{d}x}\left( t_1(x;\xi ),x \right) =f_{s}^{(1)}\frac{\mathrm{d}f_{s}^{(2)}}{\mathrm{d}x}\left( t_1(x;\xi ),x \right) .
$$
Thereby,
$$
u_s\left( t_1\left( x;\xi \right) ,x \right) =u_s(0,\xi )+\int_{\xi}^x{f_{s}^{(1)}\frac{\mathrm{d}f_{s}^{(2)}}{\mathrm{d}s}\left( t_1\left( s;\xi \right) ,s \right) \mathrm{d}s}
$$
Therefore,
$$
\left| u_s\left( t_1(x;\xi ),x \right) \right|\leqslant c\left\| \varphi \parallel _{C^1}+\left\| f^{(1)} \right\| _{C^1}\left\| f^{(2)} \right\| _{C^1} \right) 
$$
Similarly, we can get the estimate of $u_r\left( t_1(x;\xi ),x \right) $:
$$
\left| u_r\left( t_1(x;\xi ),x \right) \right|\leqslant c\left\| \varphi \parallel _{C^1}+\left\| f^{(1)} \right\| _{C^1}\left\| f^{(2)} \right\| _{C^1} \right) .
$$
Below we make inductive assumptions that for $j=1,\cdots,p-1$, the following estimates hold:
\begin{equation}\label{inductive assumption}
\left| u\left( t_j(x;\xi ),x \right) \right|\leqslant C_A\left\| \varphi \parallel _{c^1}+\parallel \psi \parallel _{c^1}+\parallel g\parallel _{c^1}+\left\| f^{(1)} \right\| _{c^1}\left\| f^{(2)} \right\| _{c^1} \right) 
\end{equation}
Then for $j=p$, we have
$$
\frac{\mathrm{d}u_s}{\mathrm{d}x}\left( t_p(x;\xi ),x \right) =f_{s}^{(1)}\frac{\mathrm{d}f_{s}^{(2)}}{\mathrm{d}x}\left( t_p(x;\xi ),x \right) .
$$
and thereby
$$
u_s\left( t_p\left( x;\xi \right) ,x \right) =u_s(t_p(0,\xi ),0)+\int_{0}^x{f_{s}^{(1)}\frac{\mathrm{d}f_{s}^{(2)}}{\mathrm{d}s}\left( t_p\left( s;\xi \right) ,s \right) \mathrm{d}s}
$$
Using \eqref{e5} and the inductive assumption, we can get
$$
u_s\left(t_p(0 ; \xi), 0\right)=A_2 u_r\left(\tilde{t}_{p-1}(0 ; \xi), 0\right)+g_s+h_s .
$$
and then
$$
\left| u_s\left( t_p(0;\xi ),0 \right) \right|\leqslant C_A\left\| \varphi \parallel _{c^1}+\parallel \psi \parallel _{c^1}+\parallel g\parallel _{c^1}+\left\| f^{(1)} \right\| _{c^1}\left\| f^{(2)} \right\| _{c^1} \right) 
$$
Therefore, \eqref{inductive assumption} holds for $j=p$.
Multiply both sides of \eqref{e4} by $b(t,x)$ and then derive for $t$,denoting $w_s$ as the derivative of $u_s$ with respect to $t$, and we have
$$
\frac{\mathrm{d}}{\mathrm{d}x}w_s+\partial _tbw_s=\partial _t\left( f_{s}^{(1)}\frac{\mathrm{d}f^{(2)}}{\mathrm{d}x} \right) 
$$
Then we can get
\begin{equation}\label{e of w_s 1}
w_s\left( t_1(x;\xi ),x \right) =w_s(0,\xi )+\int_{\xi}^x{\partial _tbw_s\mathrm{d}s}+\int_{\xi}^x{\partial _tf_{s}^{(1)}\frac{\mathrm{d}f^{(2)}}{\mathrm{d}s}\mathrm{d}s}+\left. f_{s}^{(1)}\partial _tf_{s}^{(2)} \right|_{\xi}^{x}-\int_{\xi}^x{\partial _t{f_s}^{(2)}\frac{\mathrm{d}{f_s}^{(1)}}{\mathrm{d}s}\mathrm{d}s}
\end{equation}
where
\begin{equation}\label{w_s at 0}
w_s(0, \xi)=-a \varphi^{\prime}(\xi)+f_s^{(1)}\left(\partial_t+a \partial_x\right) f_s^{(2)} .
\end{equation}
Thereby
$$
\left| w_s(0,\xi ) \right|\leqslant C\left( \left\| \varphi \right\| _{c^1}+\left\| f^{\left( 1 \right)} \right\| _{c^1}\left\| f^{(2)} \right\| _{c^1} \right) 
$$
Then we can get the estimate of $w_s\left( t_1(x;\xi ),x \right) $:
$$
\left| w_s\left( t_1(x;\xi ),x \right) \right|\leqslant \left\| \partial _tb \right\| _{c^0}\int_{\xi}^x{\left| w_s \right|\mathrm{d}s}+C\left( \left\| \varphi \right\| _{c^1}+\left\| f^{(1)} \right\| _{c^1}\left\| f^{(2)} \right\| _{c^1} \right) 
$$
Using Gronwall's inequality, we can get
$$
\left| w_s\left( t_1(x;\xi ),x \right) \right|\leqslant C_{\varepsilon _0}\left( \left\| \varphi \right\| _{c^1}+\left\| f^{(1)} \right\| _{c^1}\left\| f^{(2)} \right\| _{c^1} \right) 
$$
Below we make inductive assumptions that for $j=1,\cdots,p-1$, the following estimates hold:
\begin{equation}\label{inductive assumption for w}
\left| w\left( t_j(x;\xi ),x \right) \right|\leqslant C_{A,\varepsilon _0}\left\| \varphi \parallel _{c^1}+\parallel \psi \parallel _{c^1}+\parallel g\parallel _{c^1}+\left\| f^{(1)} \right\| _{c^1}\left\| f^{(2)} \right\| _{c^1} \right)
\end{equation}
Then for $j=p$, we have
\begin{equation}\label{e of w_s p}
	w_s\left( t_p(x;\xi ),x \right) =w_s(t_p(0,\xi ),0)+\int_{0}^x{\partial _tbw_s\mathrm{d}s}+\int_{0}^x{\partial _tf_{s}^{(1)}\frac{\mathrm{d}f^{(2)}}{\mathrm{d}s}\mathrm{d}s}+\left. f_{s}^{(1)}\partial _tf_{s}^{(2)} \right|_{0}^{x}-\int_{0}^x{\partial _t{f_s}^{(2)}\frac{\mathrm{d}{f_s}^{(1)}}{\mathrm{d}s}\mathrm{d}s}
\end{equation}
where
\begin{equation}\label{w_s at t_p(t,0)}
	w_s(t_p(0 ; \xi), 0)= A_2 w_r\left(\tilde{t}_{p-1}(0 ; \xi), 0\right)+g'_s+h'_s .
\end{equation}
and then using \eqref{inductive assumption for w} and Gronwall' inequality, we can get \eqref{inductive assumption for w} holds for $j=p$.
Finally, we can get the estimate of $u(t,x)$:
\begin{equation}\label{estimate of u}
	\left\| u(t,x) \right\|_{C^1} \leq C_{A,k,\lVert a-1 \rVert _{C^1}}(\lVert \psi \rVert _{C^1}+\lVert \varphi  \rVert _{C^1}+  \lVert g \rVert _{C^1}+\lVert f^{(1)} \rVert _{C^1}\lVert f^{(2)} \rVert _{C^1}) ,\ t\in [0,T], x\in [0,L].
\end{equation}
Below we estimate the continuity paradigm of  $h'_i(t) $ ,$\partial _x u(t,x)$and  $\partial _t u(t,x)$ .
For $h_i(t) \in \hat{H}\left[ \hat{n} \right] $ and $H_{k+1}$ ,where $t=t_p(\lambda;\xi)$for some $p \in 1,\cdots , k+1$, $\lambda \in \{ 0,L \} $, using \eqref{polynomial k+1 r}- \eqref{polynomial p s} and \eqref{d44}, we have
$$
\left|h'_i\left(t_1\right)-h'_i\left(t_2\right)\right| \leqslant C\left\| p\right\|_{C^2}(\left|x_1-x_2\right|+ \omega (\left|x_1-x_2\right|  | a(T,\cdot)))
$$
where $t_1=t_p(\lambda _1;x _1)$, $t_2=t_p(\lambda _2;x _2)$, $\lambda _1,\lambda _2 \in \{ 0,L \} $, $x _1,x _2 \in [0,L]$ and $C$ is a constant depending on $A$, $k$ and $\varepsilon _0$ and
$$
\|p\|_{c^2} \leqslant C\|h\|_{c^1} \leqslant C_{A,k,\varepsilon _0}(\lVert \psi \rVert _{C^1}+\lVert \varphi  \rVert _{C^1}+  \lVert g \rVert _{C^1}+\lVert f^{(1)} \rVert _{C^1}\lVert f^{(2)} \rVert _{C^1}) 
$$
Using \eqref{estimate of time line}, we can get
$$
\left| x_1-x_2 \right|\le C_{\left\| a-1 \right\| c_1}\left| t_p\left( \lambda ;x_1 \right) -t_p\left( \lambda ;x_2 \right) \right|=\left| t_1-t_2 \right|
$$
For $h_i(t) \in H\left[ n \right] $, using \eqref{H} and \eqref{estimate of H[n]}, we can get
$$
\left|h'_i\left(t_1\right)-h'_i\left(t_2\right)\right| \leqslant C_A(\omega \left(\left|t_1-t_2\right| \mid \psi^{\prime}\right)+\omega \left(\left|t_1-t_2\right| \mid \varphi^{\prime}\right)+\omega \left(\left|t_1-t_2\right| \mid h_i^{\prime}(t)\right)),h_i \in \hat{H}\left[ \hat{n} \right] .
$$
Thereby, we can get the continuity paradigm of $h'_i(t) $ :
$$
\omega \left( \eta \mid h_{i}^{\prime}(t) \right) \leqslant C_{A,k,\varepsilon _0}(\lVert \psi \rVert _{C^1}+\lVert \varphi  \rVert _{C^1}+  \lVert g \rVert _{C^1}+\lVert f^{(1)} \rVert _{C^1}\lVert f^{(2)} \rVert _{C^1}) \left( \omega \left( \eta \mid \varphi ^{\prime} \right) +\omega \left( \eta \mid \psi ^{\prime} \right) +\omega (\eta \mid a(T,\cdot )) \right) 
$$
Then we try to get the continuity paradigm of $\partial _x u(t,x)$ and $\partial _t u(t,x)$.We consider $\omega \left( \eta \mid \partial _tu(\cdot ,x) \right) $.
From \eqref{e of w_s} and \eqref{w_s at 0}, we know that
$$
\left| w_s\left( t_1\left( x;\xi _1 \right) ,x \right) -w_s\left( t_1\left( x;\xi _2 \right) ,x \right) \right|\leqslant C\left( \left| w_s\left( 0,\xi _1 \right) -w_s\left( 0,\xi _2 \right) \right|+	\left\| u \right\| _{C^1}\omega \left( \left| \xi _1-\xi _2 \right|\mid \partial _tb \right) +\parallel \partial _tb\parallel _{C^0}\int_0^x{\omega \left( \left| \xi _1-\xi _2 \right|\mid w_s\left( \cdot ,s \right) \right)}\mathrm{d}s+w\left( \left| \xi _1-\xi _2 \right|\mid \partial f^{(1)}\partial f^{(2)} \right) \right)
$$
where
$$
\begin{aligned}
	&\left| w_s\left( 0,\xi _1 \right) -w_s\left( 0,\xi _2 \right) \right|\leqslant \quad C\left( \omega \left( \left| \xi _1-\xi _2 \right|\mid \varphi ^{\prime} \right) +\left\| f^{(1)} \right\| _{C^1}\left\| f^{(2)} \right\| _{C^1}\left| \xi _1-\xi _2 \right| \right.\\
	&\quad +\left( \left\| \varphi \right\| _{C^1}+\left\| f^{(1)} \right\| _{C^1}\left\| f^{(2)} \right\| _{C^1} \right) \omega \left( \left| \xi _1-\xi _2 \right|\mid a \right)\\
	&\quad +\left. \left\| f^{(1)} \right\| _{C^1}\left( \omega \left( \left| \xi _1-\xi _2 \right|\mid \partial _x{f_s}^{(2)}(0,\cdot ) \right) +\omega \left( \left| \xi _1-\xi _2 \right|\mid \partial _xf_{s}^{(2)}(0,\cdot ) \right) \right) \right)\\
\end{aligned}
$$
Using Gronwall' inequality ,\eqref{time line} and \eqref{estimate of time line}, we can get
$$
\left| \xi _1-\xi _2 \right|\leqslant C\left| t_1\left( x;\xi _1 \right) -t_1\left( x;\xi _2 \right) \right|
$$
and then
$$
\begin{aligned}
	\omega \left( \left| \xi _1-\xi _2 \right|\mid w_s\left( t_1\left( x;\cdot \right) ,x \right) \right) &\leqslant \quad C\left( \omega \left( \left| \xi _1-\xi _2 \right|\mid \varphi ^{\prime} \right) +\left\| f^{(1)} \right\| _{C^1}\left\| f^{(2)} \right\| _{C^1}\left| \xi _1-\xi _2 \right|+\left\| u \right\| _{C^1}\omega \left( \left| \xi _1-\xi _2 \right|\mid \partial _tb \right) \right.\\
	&\quad +\left( \left\| \varphi \right\| _{C^1}+\left\| f^{(1)} \right\| _{C^1}\left\| f^{(2)} \right\| _{C^1} \right) \omega \left( \left| \xi _1-\xi _2 \right|\mid a \right) +\omega \left( \left| \xi _1-\xi _2 \right|\mid \partial f^{(1)}\partial f^{(2)} \right)\\
	&\quad +\left. \left\| f^{(1)} \right\| _{C^1}\left( \omega \left( \left| \xi _1-\xi _2 \right|\mid \partial _x{f_s}^{(2)}(0,\cdot ) \right) +\omega \left( \left| \xi _1-\xi _2 \right|\mid \partial _xf_{s}^{(2)}(0,\cdot ) \right) \right) \right)\\
\end{aligned}
$$
Below we make inductive assumptions that for $j=1,\cdots,p-1$, the following estimates hold:
\begin{equation}\label{inductive assumption for w eta}
\begin{aligned}
	\omega \left( \eta \mid \omega \left( t_j\left( x;\cdot \right) ,x \right) \right) &\leqslant \quad C\left( \omega \left( \eta \mid \varphi ^{\prime} \right) +\left\| f^{(1)} \right\| _{C^1}\left\| f^{(2)} \right\| _{C^1}\eta +\left\| u \right\| _{C^1}\omega \left( \eta \mid \partial _tb \right) \right.\\
	&\quad +\left( \left\| \varphi \right\| _{C^1}+\left\| f^{(1)} \right\| _{C^1}\left\| f^{(2)} \right\| _{C^1} \right) \omega \left( \eta \mid a \right) +\omega \left( \eta \mid \partial f^{(1)}\partial f^{(2)} \right)\\
	&\quad +\left. \left\| f^{(1)} \right\| _{C^1}\left( \omega \left( \eta \mid \partial _x{f_s}^{(2)}(0,\cdot ) \right) +\omega \left( \eta \mid \partial _xf_{s}^{(2)}(0,\cdot ) \right) \right) \right)\\
	\,\,+\omega \left( \eta \mid g^{\prime} \right) +\omega \left( \eta \mid h^{\prime} \right)\\
\end{aligned}
\end{equation}
Then for $j=p$, from \eqref{e of w_s p}, we have
$$
\left| w_s\left( t_p\left( x;\xi _1 \right) ,x \right) -w_s\left( t_p\left( x;\xi _2 \right) ,x \right) \right|\leqslant C\left( \left|w_s\left(t_p\left(0 ; \xi_1\right), 0\right)-w_s\left(t_p\left(0 ; \xi_2\right), 0\right)\right|+	\left\| u \right\| _{C^1}\omega \left( \left| \xi _1-\xi _2 \right|\mid \partial _tb \right) +\parallel \partial _tb\parallel _{C^0}\int_0^x{\omega \left( \left| \xi _1-\xi _2 \right|\mid w_s\left( \cdot ,s \right) \right)}\mathrm{d}s+w\left( \left| \xi _1-\xi _2 \right|\mid \partial f^{(1)}\partial f^{(2)} \right) \right)
$$
where from \eqref{w_s at t_p(t,0)}, we know that
$$
\omega \left( \eta \mid w_s\left( t_p\left( 0;\cdot \right) ,0 \right) \right) \leqslant C\left( \omega \left( \eta \mid w\left( t_{p-1}\left( x;\cdot \right) ,x \right) +\omega \left( \eta \mid g^{\prime} \right) +\omega \left( \eta \mid h^{\prime} \right) \right) \right. 
$$
Then using Gronwall's inequality and \eqref{inductive assumption for w eta}, we can get \eqref{inductive assumption for w eta} holds for $j=p$.
Next we consider $\omega\left(\eta \mid \omega\left(t_p(\cdot, \xi), \cdot\right)\right)$.
From \eqref{e of w_s 1} and \eqref{e of w_s p}, we know that
$$
\omega \left( \eta \mid w\left( t_p(\cdot ;\xi ),\cdot \right) \right) \leqslant \left( \left\| \partial _ta \right\| _{c^0}\parallel u\parallel _{c^1}+\left\| f^{(1)} \right\| _{c^1}\left\| f^{(2)} \right\| _{c^1} \right) \eta 
$$
Therefore, we take
$$
\begin{aligned}
	\varOmega (\eta )&=\quad C_0C_{A,k,\varepsilon _0}\left( \omega \left( \eta \mid \varphi ^{\prime} \right) +(\left\| f^{(1)} \right\| _{C^1}\left\| f^{(2)} \right\| _{C^1}+\left\| \partial _ta \right\| _{C^0}\left\| \psi \right\| _{C^1}+\left\| \varphi \right\| _{C^1}+\left\| g \right\| _{C^1}+\left\| f^{(1)} \right\| _{C^1}\left\| f^{(2)} \right\| _{C^1})\eta +\left\| \psi \right\| _{C^1}+\left\| \varphi \right\| _{C^1}+\left\| g \right\| _{C^1}+\left\| f^{(1)} \right\| _{C^1}\left\| f^{(2)} \right\| _{C^1}\omega \left( \eta \mid \partial _tb \right) \right.\\
	&\quad +\left( \left\| \varphi \right\| _{C^1}+\left\| f^{(1)} \right\| _{C^1}\left\| f^{(2)} \right\| _{C^1} \right) \omega \left( \eta \mid a \right) +\omega \left( \eta \mid \partial f^{(1)}\partial f^{(2)} \right) +\omega \left( \eta \mid \psi ^{\prime} \right)\\
	&\quad +\left. \left\| f^{(1)} \right\| _{C^1}\left( \omega \left( \eta \mid \partial _x{f_s}^{(2)}(0,\cdot ) \right) +\omega \left( \eta \mid \partial _xf_{s}^{(2)}(0,\cdot ) \right) \right) \right)\\
\end{aligned}
$$
where $C_0 \gg 1 $ is a constant, and then we know that
$$
\omega \left( \eta \mid h'_i(t) \right) \leqslant \varOmega (\eta ) .
$$
$$
\omega \left( \eta \mid \partial _tu(t,x) \right) \leqslant \varOmega (\eta ) .
$$
For $\partial _xu(t,x)$, from \eqref{e1} we have
$$
\omega \left( \eta \mid \partial _xu \right) \leqslant C\left( \omega \left( \eta \mid \partial _tu \right) +\parallel u\parallel _{C^1}\omega (\eta \mid a)+\omega \left( \eta \mid \partial f^{(1)}\partial f^{(2)} \right) \right) 
$$
Thereby, after adjusting $C_0$, we can get
$$
\omega \left( \eta \mid \partial _xu(t,x) \right) \leqslant \varOmega (\eta ) .
$$
\begin{rem}
	From the above analysis, we can see that the continuity paradigm of $h'_i(t) $, $\partial _xu(t,x)$ and $\partial _tu(t,x)$ is uniform in $t$ and $x$.
\end{rem}

\section{linearization iteration}























\section{Appendice}
%%%%%%%%%%%%%%%%%%%%%%%%%%%%%%%%%%%%%%%%%%%%%%%%%%%%%%%%%%%%%%%%%%%%%%%%%%%%%%%%%%%%%%%%%%%%%%%%%%%%%%
\begin{lem}\label{l50}
	For the equation
	\begin{equation}\label{51}
		\psi =A^kBH_{k+1} +\cdots +ABH_2 +BH_1.
	\end{equation}
	$\psi , H_i \in \mathbb{R} ^n ,i = 1,\cdots,k+1$,
	$$
	A=\left(\begin{array}{ll} 
		& A_1 \\
		A_2 &
	\end{array}\right),
	$$
	$$
	B=\left( \begin{matrix}
		I_{r_1}&		&		&		\\
		&		O_{r_2}&		&		\\
		&		&		I_{s_1}&		\\
		&		&		&		O_{s_2}\\
	\end{matrix} \right) ,
	$$
	where $r_1 +r_2 =r,\ s_1+s_2=s, r+s=n $, $A_1=\left( a_{ij} \right) ,\ i=1,\cdots ,r,\ j=r+1,\cdots ,n$, $A_2=\left( a_{ij} \right) ,\ i=r+1,\cdots ,n,\ j=1,\cdots ,r$.If $\psi$ satisfies the following condition
	\begin{equation}\label{52}
		\psi _i=\sum_{j=1}^r{a_{ij}\psi _j ,\,\,i=r+s_1+1,\cdots ,r+s.}
	\end{equation}
	and the Kalman rank condition holds as follows,
	\begin{equation}\label{53}
		\operatorname{Rank}\left[B, A^k B, \ldots, A^{k-1} B\right]=n,
	\end{equation}
	where $r+s=n$,then there exists a set of solutions $H_p (x), (p = 1, \cdots , k+1)$ of equation \eqref{51} satisfying
	\begin{align}
		H^{-}_{2m-1} &= H^{-}_{2m}\label{55} ,\\
		H^{+}_{2m+1} &= H^{+}_{2m}\label{56} .
	\end{align}
\end{lem}

\begin{proof}
	Without loss of generality, we can assume that $k$ is odd. The condition \ref{52} can be written as
\begin{equation}\label{513}
	\psi^{+}=A_2\psi ^{-}+\bar{H} .
\end{equation}
$\bar{H}$  is the difference term, i.e.
\begin{equation}
    \bar{H}_i =\psi _i- \sum_{j=1}^r{a_{ij}\psi _j,\,\,i=r+1,\cdots ,r+s_1}.
\end{equation}
By the definition of the matrix $B$, the $i$-th $(i=r_1+1,\cdots,r,s_1+1,\cdots,r+s)$ components of $H_p(p = 1, \cdots , k+1)$ can be taken to be zero.The following is an attempt to write conditions \eqref{55}-\eqref{56} as matrix equations. Denote
$$
H=\left( H^{-}_{k+1},H^{+}_{k+1},\cdots ,H^{-}_{1},H^{+}_{1} \right) ^T .
$$
Then the equation \eqref{51} ,\eqref{54}-\eqref{55}can be written as a matrix equation as follows:
$$
\binom{\psi}{0} = \left( \begin{matrix}
   & A_1A_2\cdots A_2&		A_1\cdots A_1&	&	\cdots&		&A_1 & I _r&		\\
    A_2A_1\cdots A_1&	&	&A_2\cdots A_2&		\cdots&		A_2&	&	&I_s\\
    I_r& &-I_r &  & & & & & \\
    & & & I_s & & & & &\\
    & & &  & \cdots& & & &\\
    & & &  & & & -I_s& &\\
    & & &  & &I_r & & -I_r& 
\end{matrix} \right) H.
$$
After appropriate matrix row transformations, the first row of the equation can be written as
$$
\psi = \left( \begin{matrix}
  &  A_1A_2\cdots A_2&		A_1\cdots A_1&	&	\cdots&	&	A_1&	I_r&	\\
   & &  A_2A_1\cdots A_1&	&	\cdots&		&	A_2A_1 &	A_2 &	I_s
\end{matrix} \right) H.
$$
Let $H^{+}_{k+1}=0$. Combined with \eqref{513}, equation \eqref{51} can be rewritten as
\begin{equation}\label{514}
	\binom{\psi ^{-}}{\bar{H} } =\left( \begin{matrix}
		&  &		A_1\cdots A_1&	&	\cdots&	&	A_1&	I_r&	\\
		 & &  &	&	\cdots&	&	 & &	I_s
	  \end{matrix} \right) H.
\end{equation}
Notice that \eqref{54} yields that the rank of the equation generalization matrix is equal to the rank of the original coefficient matrix, and thus \eqref{514} has a solution.
\end{proof}

\begin{cor}\label{l51}
	Under the conditions of the above lemma, if $\psi$ satisfies
	\begin{equation}\label{515}
		\psi _i=\sum_{j=r+1}^{r+s}{a_{ij}\psi _j ,\,\,i=r_1+1,\cdots ,r,}
	\end{equation}
	then there exists a set of solutions $H_p (x), (p = 1, \cdots , k+1)$ of equation \eqref{51} satisfying
	\begin{align}
		H^{+}_{2m-1} &= H^{+}_{2m}\label{516} ,\\
		H^{-}_{2m+1} &= H^{-}_{2m}\label{517} .
	\end{align}
\end{cor}
\begin{proof}
	Define $\tilde{\psi} $ as $\tilde{\psi}^{-}=\psi ^{+}$,$\tilde{\psi}^{}=\psi ^{-}$,$\tilde{A}$ as $\tilde{A}^{1}=A ^{2}$,$\tilde{A}^{2}=A ^{1}$, and $\tilde{H}_p$ as $\tilde{H}_p^{-}=H_p ^{+}$,$\tilde{H}_p^{+}=H_p ^{-}$, which are the same as the above lemma. 
\end{proof}

\begin{cor}\label{l52}
	In the proof of Lemma \ref{l50} one obtains $H_1^{+}=\bar{H} $, and in the conditions of Corollary \ref{l51} one obtains $H_1^{-}=\bar{H} $.
\end{cor}





\bibliographystyle{plain}
%\bibliographystyle{abbrv}
%\bibliographystyle{unsrt}
%\bibliographystyle{alpha}

\bibliography{WNGD}

%\begin{thebibliography}{99}
%
% \smallskip
%
%\bibitem{Dafermos} Dafermos C.M. \& Hsiao L.; Hyperbolic systems of balance laws with inhomogeneity and dissipation. {\em Indiana Univ. Math. J.}, {\bf 31} (1982), no. 4, 471--491.
%
%\bibitem{Glimm} Glimm J.; Solutions in the large for nonlinear hyperbolic systems of equations. {\em Comm. Pure Appl. Math.}, {\bf 18} (1965), 697--715.
%
%\bibitem{Glimm_Lax} Glimm J. \& Lax P.D.; Decay of Solutions of Systems of Nonlinear Hyperbolic Conservation Laws. {\em Mem. Amer. Math. Soc.}, {\bf 101} (1970), 1--112.
%
%\bibitem{Majda} Majda A. \& Rosales R.; Resonantly interacting weakly nonlinear hyperbolic waves, I. a single space variable. {\em Stud. Appl. Math.}, {\bf 71} (1984), no. 2, 149--179.
%
%\bibitem{Pego} Pego R.L.; Some explicit resonating waves in weakly nonlinear gas dynamics. {\em Stud. Appl. Math.}, {\bf 79} (1988), no. 3, 263--270.
%
%
%\end{thebibliography}

\end{document}
